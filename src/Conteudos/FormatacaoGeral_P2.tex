%%%%%%%%%%%%%%%%%%%%%%%%%% AULA 02 %%%%%%%%%%%%%%%%%%%%%%%%%%

%%%%%%%%%%%%%%%%%%%%%%% FINALIDADE %%%%%%%%%%%%%%%%%%%%%%%%%%
%			Demonstrar comandos sobre formatação			%
%			de tabelas e figuras, além de variações			%
%			de listagem.									%
%%%%%%%%%%%%%%%%%%%%%%% FIM FINALIDADE %%%%%%%%%%%%%%%%%%%%%%

%%%%%%%%%%% SLIDE 01 %%%%%%%%%%%%%%%%%%%%%%%%
\begin{frame}{Itens}
	\begin{Codigo}{Exemplo de itens com marcador}
		\LCmdArg{begin}{itemize} \n
			\LCmd{item} Primeiro item; \n
				\LCmdArg{begin}{itemize} \n
					\LCmd{item} Sub-item; \n
					\LCmd{item Outro} sub-item; \n
				\LCmdArg{end}{itemize} \n
			\LCmd{item} Último item. \n
		\LCmdArg{end}{itemize}
	\end{Codigo}

	\pause
	Produz:
	
		\begin{Resultado}{}
			\begin{itemize}
			\item Primeiro item;
				\begin{itemize}
					\item Sub-item;
					\item Outro sub-item;
				\end{itemize}
			\item Último item.
			\end{itemize}
		\end{Resultado}
\end{frame}
%%%%%%%%%%% FIM SLIDE 01 %%%%%%%%%%%%%%%%%%%%

%%%%%%%%%%% SLIDE 02 %%%%%%%%%%%%%%%%%%%%%%%%
\begin{frame}{Enumerados}
	\begin{Codigo}{Exemplo com numeração}
		\LCmdArg{begin}{enumerate}\n
			\LCmd{item} Primeiro;\n
			\LCmd{item} Segundo; \n
				\LCmdArg{begin}{enumerate} \n
					\LCmd{item} Sub-item; \n
					\LCmd{item} Sub-item. \n
			\LCmdArg{end}{enumerate} \n
		\LCmdArg{end}{enumerate}
	\end{Codigo}

\pause
	Produz:

	\begin{Resultado}{}
		\begin{enumerate}
			\item Primeiro;
			\item Segundo;
				\begin{enumerate}
					\item Sub-item;
					\item Sub-item.
				\end{enumerate}
		\end{enumerate}
	\end{Resultado}
\end{frame}
%%%%%%%%%%% FIM SLIDE 02 %%%%%%%%%%%%%%%%%%%%

%%%%%%%%%%% SLIDE 03 %%%%%%%%%%%%%%%%%%%%%%%%
\begin{frame}{Descrições}
	\begin{Codigo}{Exemplo de descrição}
		\LCmdArg{begin}{description} \n
			\LCmd{item} [Windows] Sistema operacional da Microsoft; \n
			\LCmd{item} [MacOS] Sistema operacional da Apple; \n
			\LCmd{item} [Linux] Sistema operacional livre. \n
		\LCmdArg{end}{description}
	\end{Codigo}

\pause
	Produz:

	\begin{Resultado}{}
		\begin{description}
			\item [Windows] Sistema operacional da Microsoft;
			\item [MacOS] Sistema operacional da Apple;
			\item [Linux] Sistema operacional livre.
		\end{description}
	\end{Resultado}
\end{frame}
%%%%%%%%%%% FIM SLIDE 03 %%%%%%%%%%%%%%%%%%%%

%%%%%%%%%%% SLIDE 04 %%%%%%%%%%%%%%%%%%%%%%%%
\begin{frame}{Construido Tabelas}
	O ambiente \Lenv{tabular} é usado para definir tabelas em modo texto (que não contenham nenhuma ou pouca matemática).
	
	\pause

	\begin{Codigo}{Sintaxe}
		\LCmdArg{begin}{tabular}\Larg{colunas}
			linhas \\
		\LCmdArg{end}{tabular}

		\LCmdArg{begin}{tabular*}\Larg{tamanho}\LOptArg[pos]{colunas}
			linhas \\
		\LCmdArg{end}{tabular*}
	\end{Codigo}
\end{frame}
%%%%%%%%%%% FIM SLIDE 04 %%%%%%%%%%%%%%%%%%%%

%%%%%%%%%%% SLIDE 05 %%%%%%%%%%%%%%%%%%%%%%%%
\begin{frame}{Ambiente \Lenv{tabular} - 0}
	\begin{description}
		\item [pos] Posicionamento vertical em relação ao texto (Detalhado melhor aqui) % Link?
		\pause
		\item [tamanho] Este argumento se aplica apenas para o ambiente %verb-tabular*-;
		\pause
		\item [colunas] Comando de formatação das colunas. Aonde é definido a posição do texto em cada coluna bem como as bordas laterais e espaçamentos.
	\end{description}
	
	\begin{description}
		\pause	
		\item [l] Conteúdo da coluna alinhado a esquerda
		\pause
		\item [c] Conteúdo da coluna alinhado ao centro
		\pause
		\item [r] Conteúdo da coluna alinhado a direita
		\pause
		\item [|] Desenha uma linha vertical
		\pause
		\item [||] Desenha duas linhas verticais , uma seguida da outra
	\end{description}
\end{frame}
%%%%%%%%%%% FIM SLIDE 05 %%%%%%%%%%%%%%%%%%%%

%%%%%%%%%%% SLIDE 06 %%%%%%%%%%%%%%%%%%%%%%%%
\begin{frame}{Ambiente \Lenv{tabular} - 1}
	\begin{description}
		\item [p\{wd\}] O texto na coluna é inserido em linha com largura \emph{wd} e a primeira linha é alinhada com as outras colunas
		\pause
		\item [@\{texto\}] Insere em cada linha o texto ou expressão
		\pause
		\item [Linhas] Cada linha deve terminar com \string\ \string\. Dentro da linha as celulas de cada coluna são separadas por \& conforme da definido antes
		\pause
		\item [\string\hline] Este comando desenha um traço horizontal depois da linha da coluna anterior e antes da subsequente
		
	\end{description}
\end{frame}
%%%%%%%%%%% FIM SLIDE 06 %%%%%%%%%%%%%%%%%%%%

%%%%%%%%%%% SLIDE 07 %%%%%%%%%%%%%%%%%%%%%%%%
\begin{frame}{Ambiente \Lenv{tabular} - 2}
	\begin{Codigo}{Exemplo}
		\LCmdArg{begin}{tabular}\Larg{p\{3cm\}|c|r}
			\LCmd{hline}\n
			Elemento   \&  Porcentagem \& Fator \string\\ \n
			\string\hline\string\hline \n
			Ferro      \&  10          \& 3     \string\\ \string\hline \n
			Cloro      \&  33          \& 7     \string\\ \string\hline \n
			Oxigênio   \&  51          \& 1     \string\\ \string\hline \n
		\LCmdArg{end}{tabular}
	\end{Codigo}

\pause
	Produz:
	
	\begin{Resultado}{}
		\begin{center}
			\begin{tabular}{p{3cm}|c|r}
				\hline
				Elemento    &  Porcentagem  & Fator \\ \hline \hline
				Ferro   &  10   & 3     \\ \hline
				Cloro   &  33   & 7     \\ \hline
				Oxigênio  &  51   & 1 \\ \hline
			\end{tabular}		
		\end{center}
	\end{Resultado}

	%\begin{Observacao}{Observação} tu viu a "!" que apareceu na tabela?
	%	As letras ``l'', ``c'' e ``r'' referem-se ao posicionamento do conteúdo nas colunas da tabela.
	%\end{Observacao}
\end{frame}
%%%%%%%%%%% FIM SLIDE 07 %%%%%%%%%%%%%%%%%%%%

%%%%%%%%%%% SLIDE 08 %%%%%%%%%%%%%%%%%%%%%%%%
\begin{frame}{Ambiente \Lenv{tabular} - 3}
	\begin{itemize}
		\item \texttt{@\{\}} na especificação do comando tabular resulta em uma divisão com espaçamento zero. Podemos usar para alinhar números pelo ponto decimal;
		\pause
		\item \LCmd{multicolumn} serve para juntar colunas da tabela.
	\end{itemize}
\end{frame}
%%%%%%%%%%% FIM SLIDE 08 %%%%%%%%%%%%%%%%%%%%

%%%%%%%%%%% SLIDE 09 %%%%%%%%%%%%%%%%%%%%%%%%
\begin{frame}{Ambiente \Lenv{tabular} - 4}
	\begin{Codigo}{Exemplo}
		\LCmdArg{begin}{tabular}\Larg{c r @\{,\} l}\n
			Expressão \& \LCmd{multicolumn}\Larg{2}\Larg{c}\Larg{Valor} \string\\ \string\hline\n
			\$\string\pi\$ \& 3 \& 1415 \string\\ \n
			\$\string\pi\string^2\$ \& 9 \& 869 \string\\ \n
			\$\string\pi\string^3\$ \& 31 \& 0062 \n
		\LCmdArg{end}{tabular}
	\end{Codigo}
	\pause
	Produz:

	\begin{Resultado}{}
		\begin{tabular}{c r @{,} l}
			Expressão & \multicolumn{2}{c}{Valor} \\ \hline
			$\pi$ & 3 & 1415 \\
			$\pi^2$ & 9 & 869 \\
			$\pi^3$ & 31 & 0062
		\end{tabular}
	\end{Resultado}
\end{frame}
%%%%%%%%%%% FIM SLIDE 09 %%%%%%%%%%%%%%%%%%%%

%%%%%%%%%%% SLIDE 10 %%%%%%%%%%%%%%%%%%%%%%%%
\begin{frame}{Referências cruzadas - 0}
	\begin{Resultado}{Referenciando seções, subseções, fórmulas, etc.}
		\begin{itemize}
			\item Para marcar: \LCmdArg{label}{marca};
			\pause
			\item Para referenciar: \LCmdArg{ref}{marca};
			\pause
			\item Para referenciar trocando o nome do link:
			\LCmdOptArg{hyperref}{marca}{texto}
			\pause
			\item Referenciando a página: \LCmdArg{pageref}{marca}.
		\end{itemize}
	\end{Resultado}

    \pause
	\begin{Observacao}{Observação}
		As referências são armazenadas no arquivo .AUX e por isto pode ser necessária mais de uma compilação para resolver as pendências.
	\end{Observacao}
\end{frame}
%%%%%%%%%%% FIM SLIDE 10 %%%%%%%%%%%%%%%%%%%%

%%%%%%%%%%% SLIDE 11 %%%%%%%%%%%%%%%%%%%%%%%%
\begin{frame}{Referências cruzadas - 1}
	\begin{Codigo}{Exemplo}
		\LCmdArg{begin}{equation}\n
		\LCmdArg{label}{eqn:integral}\n
		\string\int\ x\string\,\string\mathrm\{d\}x\n
		\LCmdArg{end}{equation}\n
		A equação~(\LCmdArg{ref}{eqn:integral}) define \string\dots
	\end{Codigo}

    \pause
	Produz:
	
	\begin{Resultado}{}
		\begin{equation}
			\label{eqn:integral}
			\int x\,\mathrm{d}x
		\end{equation}
		A equação~(\ref{eqn:integral}) define \dots
	\end{Resultado}
\end{frame}
%%%%%%%%%%% FIM SLIDE 11 %%%%%%%%%%%%%%%%%%%%

%%%%%%%%%%% SLIDE 12 %%%%%%%%%%%%%%%%%%%%%%%%
\begin{frame}{Citações}
	\begin{Codigo}{Exemplo}
		Bilbo costumava dizer:\n
		\LCmdArg{begin}{quote}\n
			É perigoso sair porta afora, Frodo. Você pisa na Estrada, e, se não controlar seus pés, não há como saber até onde você pode ser levado \ldots\n
		\LCmdArg{end}{quote}
	\end{Codigo}

    \pause
	Produz:

	\begin{Resultado}{}
		Bilbo costumava dizer:
		\begin{quote}\normalfont
			É perigoso sair porta afora, Frodo. Você pisa na Estrada, e, se não controlar seus pés, não há como saber até onde você pode ser levado \ldots
		\end{quote}
	\end{Resultado}
\end{frame}
%%%%%%%%%%% FIM SLIDE 12 %%%%%%%%%%%%%%%%%%%%

%%%%%%%%%%% SLIDE 13 %%%%%%%%%%%%%%%%%%%%%%%%
\begin{frame}{Versos - 0}
	\begin{Codigo}{Exemplo de versos}
		Esta é uma poesia sem sentido retirada de
		`{}`Alice Através do Espelho'{}':
		\nn
		\LCmdArg{begin}{center}\n
			\LCmdArg{textbf}{Pargarávio}\n
		\LCmdArg{end}{center}\n
		\LCmdArg{begin}{verse}\n
			Solumbrava, e os lubriciosos touvos \string\\ \n
			Em vertigiros persondavam as verdentes; \string\\ \n
			Trisciturnos calavam-se os gaiolouvos \string\\ \n
			E os porverdidos estriguilavam fientes.\n
		\LCmdArg{end}{verse}
	\end{Codigo}
\end{frame}
%%%%%%%%%%% FIM SLIDE 13 %%%%%%%%%%%%%%%%%%%%

%%%%%%%%%%% SLIDE 14 %%%%%%%%%%%%%%%%%%%%%%%%
\begin{frame}{Versos - 1}
	Produz:

	\begin{Resultado}{}
		Esta é uma poesia sem sentido retirada de
		``Alice Através do Espelho'':
		\begin{center}
			\textbf{Pargarávio}
		\end{center}
		\begin{verse}
			Solumbrava, e os lubriciosos touvos \\
			Em vertigiros persondavam as verdentes; \\
			Trisciturnos calavam-se os gaiolouvos \\
			E os porverdidos estriguilavam fientes.
		\end{verse}
	\end{Resultado}
\end{frame}
%%%%%%%%%%% FIM SLIDE 14 %%%%%%%%%%%%%%%%%%%%

%%%%%%%%%%% SLIDE 15 %%%%%%%%%%%%%%%%%%%%%%%%
\begin{frame}{Figuras e tabelas - 0}
	São \emph{corpos flutuantes} obtidos usando-se os ambientes:
	\begin{Codigo}{Figuras e Tabelas}
		\LCmdArg{begin}{figure}\Lopt[especificação] \n
			... \n
			\LCmdArg{caption}{texto} \n
		\LCmdArg{end}{figure} \n
		e \n
		\LCmdArg{begin}{table}\Lopt[especificação] \n
			... \n
			\LCmdArg{caption}{texto} \n
		\LCmdArg{end}{table}
	\end{Codigo}

	\begin{Observacao}{Observação}
		\LCmdArg{caption}{\dots} serve para incluir uma legenda.
	\end{Observacao}
\end{frame}
%%%%%%%%%%% FIM SLIDE 15 %%%%%%%%%%%%%%%%%%%%

%%%%%%%%%%% SLIDE 16 %%%%%%%%%%%%%%%%%%%%%%%%
\begin{frame}{Figuras e tabelas - 1}
	A especificação pode ser um ou mais dos seguintes (não será necessariamente seguido pelo \LaTeX):
	\pause
	\begin{description}
		\item [h] aqui;
		\pause
		\item [t] alto da página;
		\pause
		\item [b] embaixo da página;
        \pause
		\item [p] página especial;
		\pause
		\item [!] não considera alguns parâmetros internos.
	\end{description}

    \pause
	\begin{Observacao}{}
		A ordem em que são usados é relevante -- maior prioridade é dada ao primeiro e menor ao último.
	\end{Observacao}
\end{frame}
%%%%%%%%%%% FIM SLIDE 16 %%%%%%%%%%%%%%%%%%%%

%%%%%%%%%%% SLIDE 17 %%%%%%%%%%%%%%%%%%%%%%%%
\begin{frame}{Figuras e tabelas - 2}
	\fontsize{10}{11}\selectfont
	\begin{Codigo}{Exemplo}
		\LCmdArg{begin}{table}\Lopt[!tp] \n
			\LCmdArg{caption}{Tabela sem sentido} 
			\LCmdArg{label}{tab:semsentido} \n
			\LCmd{centering} \n
			\LCmdArg{begin}{tabular}\Larg{l|l} \string\hline \n
				Parâmetro \& Valor \string\\ \string\hline\string\hline \n
				XYZ \& 123 \string\\ \n
				ABC \& 321 \string\\ \string\hline \n
			\LCmdArg{end}{tabular} \n
		\LCmdArg{end}{table} \n
		A Tabela\string~\LCmdArg{ref}{tab:semsentido} apresenta \string\dots
	\end{Codigo}

    \pause
	\begin{Observacao}{Observações}
		\begin{itemize}
			\item \texttt{\string\centering} serve para centralizar o tabular;
			\item comando \texttt{\string\caption\{\dots\}} usado acima do tabular devido a ABNT;
			\item comando \texttt{\string\label\{\dots\}} deve ser usado após o comando \texttt{\string\caption\{\dots\}}.
		\end{itemize}
	\end{Observacao}
\end{frame}
%%%%%%%%%%% FIM SLIDE 17 %%%%%%%%%%%%%%%%%%%%

%%%%%%%%%%% SLIDE 18 %%%%%%%%%%%%%%%%%%%%%%%%
\begin{frame}{Figuras e tabelas - 3}
	\fontsize{10}{11}\selectfont
	Produz:
	\begin{Resultado}{}
		\begin{table}
			\label{tab:semsentido}
			\caption{Tabela sem sentido}% 
			\centering
			\begin{tabular}{l|l} \hline
				Parâmetro & Valor \\ \hline\hline
				XYZ & 123 \\
				ABC & 321 \\ \hline
			\end{tabular}
		\end{table}
		A Tabela~\ref{tab:semsentido}~apresenta \ldots
	\end{Resultado}
\end{frame}
%%%%%%%%%%% FIM SLIDE 18 %%%%%%%%%%%%%%%%%%%%

%%%%%%%%%%% SLIDE 19 %%%%%%%%%%%%%%%%%%%%%%%%
\begin{frame}{Importando imagens - 0}
	O programa compilador \prog{pdftex}, usado nas atuais versões de \LaTeX{}, pode importar imagens nos formatos: JPG, PNG, PDF, MPS e EPS.
	\pause
	\begin{itemize}
		\item \LCmdArg{usepackage}{graphicx};
		\pause
		\item \LOA includegraphics[especificação]{nome do arquivo sem extensão};
		\item[] Especificação:
			\begin{description}
			    \pause
				\item [width] largura;
				\pause
				\item [height] altura;
				\pause
				\item [angle] rotaciona a figura;
			\end{description}
	\end{itemize}
\end{frame}
%%%%%%%%%%% FIM SLIDE 19 %%%%%%%%%%%%%%%%%%%%

%%%%%%%%%%% SLIDE 20 %%%%%%%%%%%%%%%%%%%%%%%%
\begin{frame}{Importando imagens - 1}
	\begin{Codigo}{Exemplo}
		\LCmdArg{documentclass}{article}\n
		...\n
		\LCmdArg{usepackage}{graphicx}\n
		\LCmdArg{begin}{document}\n
			...\n
			\LCmdArg{begin}{figure}\LOpt{!tp}\n
				\LCmd{centering}\n
				\LOA includegraphics[width=0.6\LCmd{textwidth}]{<nome imagem>}\n
				\LCmdArg{caption}{\dots}\LCmdArg{label}{chave}\n
			\LCmdArg{end}{figure}\n
			...\n
		\LCmdArg{end}{document}
	\end{Codigo}
\end{frame}
%%%%%%%%%%% FIM SLIDE 20 %%%%%%%%%%%%%%%%%%%%

%%%%%%%%%%% SLIDE 21 %%%%%%%%%%%%%%%%%%%%%%%%
\begin{frame}{Produzindo sumários}
	Estes podem ser obtidos pelos comandos: 
	\begin{itemize}
	    \pause
		\item \LCmd{tableofcontents},
		\pause
		\item \LCmd{listoffigures},
		\pause
		\item \LCmd{listoftables}.
	\end{itemize}
\end{frame}
%%%%%%%%%%% FIM SLIDE 21 %%%%%%%%%%%%%%%%%%%%

%%%%%%%%%%% SLIDE 22 %%%%%%%%%%%%%%%%%%%%%%%%
\begin{frame}{Estrutura geral}
	\begin{Codigo}{Estrutura de um artigo com sumários}
		\LCmdArg{documentclass}{article}\n
		...\n
		\LCmdArg{begin}{document}\n
			\LCmd{maketitle}\n
			\LCmd{tableofcontents}\n
			\LCmd{listoffigures}\n
			\LCmd{listoftables}\n
			\LCmdArg{section}{Introdução}\n
			...\n
		\LCmdArg{end}{document}
	\end{Codigo}


    \pause
	\begin{Observacao}{Observação}
		São produzidos os arquivos .TOC, .LOF e .LOT. Posteriormente eles podem ser editados.
	\end{Observacao}
\end{frame}
%%%%%%%%%%% FIM SLIDE 22 %%%%%%%%%%%%%%%%%%%%

%%%%%%%%%%% SLIDE 23 %%%%%%%%%%%%%%%%%%%%%%%%
\begin{frame}{Cores - 0}
	Para usar cores é necessário o uso de alguns pacotes adicionais:
	\pause
	\begin{Codigo}{}
		\LCmdArg{usepackage}{color}\n
		ou\n
		\LCmdOptArg{usepackage}{usenames,dvipsnames,svgnames,table}{xcolor}
	\end{Codigo}
	
	\pause

	Assim podemos definir algumas cores básicas como \textcolor{blue}{azul},\textcolor{green}{verde} ou até  mesmo \textcolor{pink}{Rosa} para os elementos do \LaTeX
\end{frame}
%%%%%%%%%%% FIM SLIDE 23 %%%%%%%%%%%%%%%%%%%%

%%%%%%%%%%% SLIDE 24 %%%%%%%%%%%%%%%%%%%%%%%%
\begin{frame}{Cores - 1}
	Assim como muitos comandos, podemos indicar as cores de duas formas básicas:
	\pause
	\begin{Codigo}{}
		\LCmdArg{textcolor}{nome-cor}\Larg{algum texto}\n ou\n \Larg{\LCmdArg{color}{nome-cor} algum texto}.
	\end{Codigo}

    \pause
	Caso não tenha a cor exata definida pode-se ainda definir uma cor personalizada usando o comando \footnote{Mais detalhes em \url{http://www.las.ic.unicamp.br/pub/ctan/macros/latex/contrib/xcolor/xcolor.pdf}}:
	
	\pause
	\begin{Codigo}{}
		\LCmdArg{definecolor}{''name''}\Larg{''model''}\Larg{''color-spec''}
	\end{Codigo}
\end{frame}
%%%%%%%%%%% FIM SLIDE 24 %%%%%%%%%%%%%%%%%%%%

%%%%%%%%%%% SLIDE 25 %%%%%%%%%%%%%%%%%%%%%%%%
\begin{frame}{Modelos de Cores - 0}
	\begin{Observacao}{}
		\begin{center}
			\begin{tabular}{cp{3cm}cl}
				\textbf{Modelo} & \textbf{Descrição} & Variação Paramentro\\
				\hline
				gray & Tons de cinza & 0-1 \\
				RGB & Vermelho, Verde, Azul & 0-255 \\
				HTML & Vermelho, Verde, Azul & 00-FF \\
				cmyk & Ciano, Magenta, Amarelo e Preto & 0-1 \\
			\end{tabular}
		\end{center}
	\end{Observacao}
\end{frame}
%%%%%%%%%%% FIM SLIDE 25 %%%%%%%%%%%%%%%%%%%%

%%%%%%%%%%% SLIDE 26 %%%%%%%%%%%%%%%%%%%%%%%%
\begin{frame}{Modelos de Cores - 1}
	\begin{Observacao}{}
		\begin{center}
			\begin{tabular}{cl}
				\textbf{Modelo} & \textbf{Exemplo}\\
				\hline
				gray & \LCmdArg{definecolor}{light-gray}\Larg{gray}\Larg{0.95}\\
				rgb & \LCmdArg{definecolor}{orange}\Larg{rgb}\Larg{1,0.5,0}\\
				RGB & \LCmdArg{definecolor}{orange}\Larg{rgb}\Larg{255,127,0}\\
				HTML & \LCmdArg{definecolor}{orange}\Larg{rgb}\Larg{FF7F00}\\
				cmyk & \LCmdArg{definecolor}{orange}\Larg{cmyk}\Larg{0,0.5,1.0}\\
			\end{tabular}
		\end{center}
	\end{Observacao}
\end{frame}
%%%%%%%%%%% FIM SLIDE 26 %%%%%%%%%%%%%%%%%%%%

%%%%%%%%%%% SLIDE 27 %%%%%%%%%%%%%%%%%%%%%%%%
\begin{frame}{Modos do \TeX}
	\begin{description}
	    \pause
		\item [Modo parágrafo] Divide texto em linhas, parágrafos e páginas; é o modo normal do \TeX;
		\pause
		\item [Modo LR] Descarrega os tipos sem dividir texto; obtido usando-se \LCmdArg{mbox}{} (\LCmd{mbox} pode ser usado quando não desejamos que uma palavra seja dividida em duas linhas/páginas, por exemplo, \LCmdArg{mbox}{555-1234});
		\pause
		\item [Modo matemático] Para produzir fórmulas matemáticas:
		\pause
			\begin{Codigo}{}
				\texttt{\bs(\dots\bs)}\n
				\texttt{\$\dots\$}\n
				\LCmdArg{begin}{displaymath} \dots \LCmdArg{end}{displaymath}\n
				\texttt{\bs\ls\dots\bs\rs}\n
				\LCmdArg{begin}{equation} \dots \LCmdArg{end}{equation}\n
				\LCmdArg{begin}{eqnarray} \dots \LCmdArg{end}{eqnarray}.
			\end{Codigo}
	\end{description}
\end{frame}
%%%%%%%%%%% FIM SLIDE 27 %%%%%%%%%%%%%%%%%%%%

%%%%%%%%%%%%%%%%%%%%%%%%%% FIM AULA 02 %%%%%%%%%%%%%%%%%%%%%%