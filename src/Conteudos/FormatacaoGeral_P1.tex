%%%%%%%%%%%%%%%%%%%%%%%%%% AULA 01 %%%%%%%%%%%%%%%%%%%%%%%%%%

%%%%%%%%%%%%%%%%%%%%%%% FINALIDADE %%%%%%%%%%%%%%%%%%%%%%%%%%
%			Tratar da estrutura dos comandos em LaTeX   	%
%			e TeX, apontando os principais recursos para	%
%			diagramação de documentos, além de demonstrar	%
%			comandos sobre formatação geral de documentos	%
%%%%%%%%%%%%%%%%%%%%%%% FIM FINALIDADE %%%%%%%%%%%%%%%%%%%%%%

%%%%%%%%%%% SLIDE 10 %%%%%%%%%%%%%%%%%%%%%%%%%
\begin{frame}{Os comandos do \LaTeX}
	\begin{itemize}
		\item Os comandos são necessários para que \LaTeX\ possa formatar o texto (\LaTeX\ não é tão inteligente como um designer/tipógrafo humano);
		\pause
		\item Os comandos \TeX\ normalmente são antecedidos de ``\texttt{\textbackslash}'' (por exemplo, para obter \LaTeX\ deve-se digitar \LCmd{LaTeX} e para obter ``\textbackslash'' deve-se digitar \texttt{\$}\LCmd{backslash}\texttt{\$} ou \LCmd{textbackslash});
		\pause
		\item A linguagem \TeX\ segue as regras/ideias de linguagens de programação (declarações e corpo do programa; ligação de bibliotecas; regras de escopo; etc.);
	\end{itemize}
	
    \begin{Observacao}{Observação}
	    Maiúsculas $\neq$ minúsculas.
    \end{Observacao}
\end{frame}
%%%%%%%%%%% FIM SLIDE 10 %%%%%%%%%%%%%%%%%%%%%

%%%%%%%%%%% SLIDE 11 %%%%%%%%%%%%%%%%%%%%%%%%%
\begin{frame}[fragile=singleslide]{Compilando, visualizando e imprimindo}
	Comandos para o Terminal do Linux:
	\begin{itemize}
		\item Compilação: \verb'$ pdflatex teste.tex' (para compilar, por exemplo, o arquivo \verb'teste.tex');
		\pause
		\item *Visualização: \verb'$ xdg-open teste.pdf' (o arquivo é recarregado automaticamente a cada modificação). Em alguns programas o resultado em .PDF aparece direitamente numa segunda janela;
		\pause
		\item Convertendo para html: \verb'$ latex2html teste.tex' (necessita instalar o comando);
		\pause
		\item Imprimindo: \verb'$ dvips teste.dvi' ou \verb'$ lpr teste.ps'.
	\end{itemize}
\end{frame}
%%%%%%%%%%% FIM SLIDE 11 %%%%%%%%%%%%%%%%%%%%%

%%%%%%%%%%% SLIDE 12 %%%%%%%%%%%%%%%%%%%%%%%%%
\begin{frame}{Estrutura e comandos \LaTeX}
	\begin{Codigo}{Estrutura geral}
		\LCmdOptArg{documentclass}{opcionais}{classe}\n
			~declarações~\dots\n
		\LCmdArg{begin}{document}\n
			~documento~\dots\n
		\LCmdArg{end}{document}
	\end{Codigo}

\medskip

	\begin{Codigo}{Para trabalhar com arquivos grandes}
		\LCmdArg{include}{nomearquivo}
			\% inclui comandos de um arquivo\n \% gera nova página antes\nn
		\LCmdArg{input}{nomearquivo} 
			\% inclui comandos de um arquivo\n \% não gera nova página
	\end{Codigo}
\end{frame}
%%%%%%%%%%% FIM SLIDE 12 %%%%%%%%%%%%%%%%%%%%%

%%%%%%%%%%% SLIDE 13 %%%%%%%%%%%%%%%%%%%%%%%%%
\begin{frame}{Estrutura dos comandos}
	\begin{itemize}
		\item Comandos \LaTeX{} são normalmente precedidos por \texttt{\textbackslash} e seguidos de parâmetros opcionais (delimitados por ``\texttt{[}`` e ``\texttt{]}'') e/ou parâmetros obrigatórios (delimitados por ``\texttt{\lb}'' e ``\texttt{\rb}'');
		\pause
		\begin{Codigo}{Exemplos}
			\LCmd{TeX}\n
			\LCmd{LaTeX}\n
			\LCmdArg{documentclass}{book}\n
			\LCmdOptArg{documentclass}{12pt}{article}\n
			\LCmdArg{begin}{document}
		\end{Codigo}
\bigskip
		\pause
		\item Uma excessão a esta regra é ``\texttt{\$}'' que delimita o ambiente matemático. Exemplo: \texttt{\$3+2\LCmdArg{sqrt}{2}\$}, que produz $3+2\sqrt{2}$.
	\end{itemize}
\end{frame}
%%%%%%%%%%% FIM SLIDE 13 %%%%%%%%%%%%%%%%%%%%%

%%%%%%%%%%% SLIDE 14 %%%%%%%%%%%%%%%%%%%%%%%%%
\begin{frame}{Espaços após um comando \TeX}
	Espaços após um comando serão consumidos até encontrar um caracter diferente de branco, resultando que
	\pause
		\begin{Codigo}{}
			\LCmd{TeX} é legal!
		\end{Codigo}
		
	\pause
	Produz:
	
		\begin{Resultado}{}
			\TeX é legal!
		\end{Resultado}
	\pause
	Para evitar isto, use \texttt{\textbackslash\textvisiblespace}\footnote{O símbolo \texttt{\textvisiblespace}  serve para representar o espaço no texto fonte.} ou \texttt{\{\}}, que interrompe o consumo de espaços em branco, ou \texttt{\textasciitilde} (espaço em branco indivisível):

	\pause
	\begin{Codigo}{}
		\LCmd{TeX}\textbackslash\textvisiblespace é legal!\n
		ou\n
		\LCmd{TeX}\{\}\textvisiblespace é legal!\n
		ou\n
		\LCmd{TeX}\textasciitilde é legal!
	\end{Codigo}
\end{frame}
%%%%%%%%%%% FIM SLIDE 14 %%%%%%%%%%%%%%%%%%%%%

%%%%%%%%%%% SLIDE 15 %%%%%%%%%%%%%%%%%%%%%%%%%
\begin{frame}{Sobre espaçamento}\fontsize{10}{12}\selectfont
	\begin{itemize}
		\item Para produzir espaço no texto pode-se usar ``\LCmd{\textvisiblespace}'', que representa o espaço simples;
		\pause
		\item Para produzir espaço negativo: \texttt{\string\!};
		\pause
		\item ``\texttt{\textasciitilde}'' produz um espaço que não pode ser dividido em uma quebra de linha; por exemplo: \texttt{fone:\ (61)\textasciitilde5551234};
		\pause
		\item \TeX\ assume que sentenças terminam com ``.'', introduzindo um espaço adicional ao final da frase. O comando \LCmd{frenchspacing} desabilita este espaço adicional;
		\pause
		\item Para obter espaço vertical: \LCmdArg{vspace}{espaço} (não permite obter espaço no início de uma página) e \LCmdArg{vspace*}{espaço} (conserva o espaço no início de uma página);
		\pause
		\item \LCmdArg{hspace}{espaço} permite obter espaço horizontal dentro de uma linha;
		\pause
		\item Pode-se usar as dimensões em pontos (pt), polegadas (in), milímetros (mm), centímetros (cm) etc.
	\end{itemize}
\end{frame}
%%%%%%%%%%% FIM SLIDE 15 %%%%%%%%%%%%%%%%%%%%%

%%%%%%%%%%% SLIDE EXTRA %%%%%%%%%%%%%%%%%%%%%%
\begin{frame}{Conversão de medidas}
    \tiny
    \begin{center}\begin{tabular} % Tabela da NASA, gerada automaticamente pelo LaTeX
      {>{\def\colunit{pt}}l<{\convertto{\rowunit}{1\colunit}}
       >{\def\colunit{mm}}l<{\convertto{\rowunit}{1\colunit}}
       >{\def\colunit{cm}}l<{\convertto{\rowunit}{1\colunit}}
       >{\def\colunit{ex}}l<{\convertto{\rowunit}{1\colunit}}
       >{\def\colunit{bp}}l<{\convertto{\rowunit}{1\colunit}}
       >{\def\colunit{pc}}l<{\convertto{\rowunit}{1\colunit}}
       >{\def\colunit{in}}l<{\convertto{\rowunit}{1\colunit}}
       >{\bfseries}l}
       \hline
    \multicolumn{1}{l}{\bfseries 1pt} & \multicolumn{1}{l}{\bfseries 1mm} & \multicolumn{1}{l}{\bfseries 1cm} & \multicolumn{1}{l}{\bfseries 1ex} & \multicolumn{1}{l}{\bfseries 1bp} & \multicolumn{1}{l}{\bfseries 1pc} & \multicolumn{1}{l}{\bfseries 1in} & \\
    \gdef\rowunit{pt} & & & & & & & \rowunit\\
    \gdef\rowunit{mm} & & & & & & & \rowunit\\
    \gdef\rowunit{cm} & & & & & & & \rowunit\\
    \gdef\rowunit{ex} & & & & & & & \rowunit\\
    \gdef\rowunit{bp} & & & & & & & \rowunit\\
    \gdef\rowunit{pc} & & & & & & & \rowunit\\
    \gdef\rowunit{in} & & & & & & & \rowunit\\
    \hline
    \end{tabular}\end{center}
\end{frame}
%%%%%%%%%%% FIM SLIDE EXTRA %%%%%%%%%%%%%%%%%%

%%%%%%%%%%% SLIDE 16 %%%%%%%%%%%%%%%%%%%%%%%%%
\begin{frame}{Delimitação de parágrafos}
	Uma ou mais linhas em branco delimita os parágrafos:
	\pause
	\begin{Codigo}{Exemplo}
		Este é o\textvisiblespace\textvisiblespace\textvisiblespace\textvisiblespace primeiro \n parágrafo.\n

		E este é o segundo!
	\end{Codigo}

	\pause
	Produz:
	\begin{Resultado}{}
		Este é o         primeiro
		parágrafo.

		E este é o segundo!
	\end{Resultado}
\end{frame}
%%%%%%%%%%% FIM SLIDE 16 %%%%%%%%%%%%%%%%%%%%%

%%%%%%%%%%% SLIDE 17 %%%%%%%%%%%%%%%%%%%%%%%%%
\begin{frame}{Comentários no arquivo fonte}
	Comentários em \TeX\ são obtidos usando-se \texttt{\%}.
	\pause
	Exemplo:
	\begin{Codigo}{Arquivo fonte com comentários}
		Este é um exemplo\n
		\% comentários são considerados\n
		\% espaços em branco\n
		de uso de comentários. \% fim do exemplo
	\end{Codigo}

	\pause
	Produz:
	\begin{Resultado}{}
		Este é um exemplo
		% comentários são considerados
		% espaços em branco
		de uso de comentários. % fim do exemplo
	\end{Resultado}
\end{frame}
%%%%%%%%%%% FIM SLIDE 17 %%%%%%%%%%%%%%%%%%%%%

%%%%%%%%%%% SLIDE 18 %%%%%%%%%%%%%%%%%%%%%%%%%
\begin{frame}{Classes disponíveis}
	\begin{itemize}
		\item Principais classes disponíveis:
		\pause
		\begin{description}
			\item [\Lcls{article}] Artigos curtos;
			\pause			
			\item [\Lcls{report}] Artigos mais longos, monografias, relatórios;
			\pause
			\item [\Lcls{book}] Livros;
		\end{description}

\medskip

		\pause
		\item Principais opções: 

\medskip

		\begin{itemize}
			\pause
			\item \Lopt{11pt} -- fonte de 11 pontos;
			\pause
			\item \Lopt{12pt} -- fonte de 12 pontos;
			\pause
			\item \Lopt{twoside} -- imprime em ambos os lados da página;
			\pause
			\item \Lopt{twocolumn} -- produz saída em duas colunas.
		\end{itemize}

\medskip
		\pause
		\item Lembre-se: \LCmdOptArg{documentclass}{opções}{classe}
	\end{itemize}
\end{frame}
%%%%%%%%%%% FIM SLIDE 18 %%%%%%%%%%%%%%%%%%%%%

%%%%%%%%%%% SLIDE 19 %%%%%%%%%%%%%%%%%%%%%%%%%
\begin{frame}{Estilos de página}
	\begin{Codigo}{}
		\LCmdArg{pagestyle}{estilo}\n
		ou\n
		\LCmdArg{thispagestyle}{estilo}
	\end{Codigo}

	\pause
	Estilos disponíveis:

	\begin{description}
		\item [plain] número de página centralizado no rodapé;
		\pause
		\item [headings] capítulo corrente e número de página no cabeçalho;
		\pause
		\item [empty] cabeçalho e rodapé vazios;
	\end{description}
\end{frame}
%%%%%%%%%%% FIM SLIDE 19 %%%%%%%%%%%%%%%%%%%%%

%%%%%%%%%%% SLIDE 20 %%%%%%%%%%%%%%%%%%%%%%%%%
\begin{frame}{Ambientes}
	O \LaTeX{} trabalha com \emph{ambientes}; o escopo de um ambiente é definido pelos comandos \LCmdArg{begin}{\dots} e \LCmdArg{end}{\dots}. \pause Exemplos:

	\begin{Codigo}{}
		\LCmdArg{begin}{document}
		...
		\LCmdArg{end}{document}
	\end{Codigo}

	\pause e

	\begin{Codigo}{}
		\LCmdArg{begin}{center}
		...
		\LCmdArg{end}{center}
	\end{Codigo}
\end{frame}
%%%%%%%%%%% FIM SLIDE 20 %%%%%%%%%%%%%%%%%%%%%

%%%%%%%%%%% SLIDE 21 %%%%%%%%%%%%%%%%%%%%%%%%%
\begin{frame}{Exemplo de um arquivo .TEX simples}
	\begin{Codigo}{Exemplo de arquivo .TEX}
		\LCmdOptArg{documentclass}{12pt}{article}\n
		\LCmdArg{begin}{document}\n
			Oi, mundo!\n
			Eu sou \LCmd{LaTeX}!\n
		\LCmdArg{end}{document}
	\end{Codigo}

\pause
que produz na saída:

	\begin{Resultado}{}
		Oi, mundo!\n
		Eu sou \LaTeX!
	\end{Resultado}
\end{frame}
%%%%%%%%%%% FIM SLIDE 21 %%%%%%%%%%%%%%%%%%%%%

%%%%%%%%%%% SLIDE 22 %%%%%%%%%%%%%%%%%%%%%%%%%
\begin{frame}{Usando pacotes}
	\begin{itemize}
		\pause
		\item Amplia as funcionalidades do \LaTeX;
		\pause
		\item Modularidade;
		\pause
		\begin{Codigo}{Exemplo}
			\LCmdArg{documentclass}{article}\n
			\LOA usepackage[brazilian]{babel}\n
			\LOA usepackage[latin1]{inputenc}\n
			\LOA usepackage[T1]{fontenc}\n
			\LCmdArg{usepackage}{lmodern}\n
			\LCmdArg{usepackage}{graphicx}\n
			\LCmdArg{usepackage}{amsmath,amssymb}\n
			\LCmdArg{usepackage}{indentfirst}\n
			\LCmdArg{usepackage}{url}\n
			\LCmdArg{begin}{document}\n
			\dots\n
			\LCmdArg{end}{document}
		\end{Codigo}
	\end{itemize}
\end{frame}
%%%%%%%%%%% FIM SLIDE 22 %%%%%%%%%%%%%%%%%%%%%

%%%%%%%%%%% SLIDE 23 %%%%%%%%%%%%%%%%%%%%%%%%%
\begin{frame}{Usando pacotes}\fontsize{10}{11}\selectfont
	\begin{description}
		\pause
		\item [babel] determina a língua usada no texto (\Lopt{brazilian}  é o português com as variantes brasileiras);
		\pause
		\item [inputenc] determina a codificação usada (use  \Lopt{latin1} no Linux, \Lopt{ansinew} no Windows e \Lopt{utf8} para a codificação universal UNICODE);
		\pause
		\item [fontenc] determina a codificação das fontes usados na saída; para o português é importante usar a codificação \Lopt{T1};
		\pause
		\item [lmodern] escolhe uma fonte vetorial com a codificação \Lopt{T1} (melhora a qualidade das fontes no PDF);
		\pause
		\item [graphicx] permite incorporar imagens no texto (formatos PDF, JPG, PNG, MPS e EPS);
		\pause
		\item [amsmath e amssymb] fontes e símbolos  matemáticos adicionais da AMS;
		\pause
		\item [indentfirst] indentação em início do primeiro parágrafo de seção;
		\pause
		\item [url] permite colocar urls no texto usando o comando \LCmdArg{url}{http://\dots}.
	\end{description}
\end{frame}
%%%%%%%%%%% FIM SLIDE 23 %%%%%%%%%%%%%%%%%%%%%

%%%%%%%%%%% SLIDE 24 %%%%%%%%%%%%%%%%%%%%%%%%%
\begin{frame}{Definindo divisões do texto}
	\LaTeX\ gera automaticamente a numeração das seções, existindo os seguintes comandos para a sua numeração:
	\pause
	\begin{Resultado}{Hierarquia}
		\begin{enumerate}
			\item \LCmd{part}
			\item \LCmd{chapter}
			\item \LCmd{section}
			\item \LCmd{subsection}
			\item \LCmd{subsubsection}
			\item \LCmd{paragraph}
			\item \LCmd{subparagraph}
		\end{enumerate}

	\end{Resultado}

	\pause
	\begin{Observacao}{}
		A classe \Lcls{article} não permite o comando \LCmd{chapter}.
	\end{Observacao}
\end{frame}
%%%%%%%%%%% FIM SLIDE 24 %%%%%%%%%%%%%%%%%%%%%

%%%%%%%%%%% SLIDE 25 %%%%%%%%%%%%%%%%%%%%%%%%%
\begin{frame}{Divisões do texto}
	\begin{Codigo}{Exemplo}
		\LCmdArg{documentclass}{article}\n
		\LOA usepackage[brazilian]{babel} \LOA usepackage[utf8]{inputenc}\n
		\LOA usepackage[T1]{fontenc} \LCmdArg{usepackage}{lmodern}\n
		\LCmdArg{begin}{document}\n
		\LCmdArg{section}{Introdução}\n
			bla, bla, bla\n
		\LCmdArg{section}{Usando o \LCmd{LaTeX}}\n
		\LCmdArg{subsection}{Uso Básico}\n
			bla, bla, bla\n
		\LCmdArg{subsection}{Uso Avançado}\n
		\LCmdArg{section}{Conclusão}\n
			bla, bla, bla\n
		\LCmdArg{end}{document}
	\end{Codigo}
\end{frame}
%%%%%%%%%%% FIM SLIDE 25 %%%%%%%%%%%%%%%%%%%%%

%%%%%%%%%%% SLIDE 26 %%%%%%%%%%%%%%%%%%%%%%%%%
\begin{frame}{Símbolos especiais}
	Os sete seguintes símbolos especiais podem ser facilmente obtidos pelos seguintes comandos:

	\pause
	\begin{center}
		\begin{tabular}{r*6c}
			\$ &\& &\% &\# &\_ &\{ &\} \\
			\LCmd{\$} &\LCmd{\&} &\LCmd{\%} &\LCmd{\#} &\LCmd{\textunderscore} &\LCmd{\lb} &\LCmd{\rb}
		\end{tabular}
	\end{center}

		\pause
		Esses símbolos são especiais porque são usados em comandos na sintaxe de \LaTeX{} e não podem ser obtidos direitamente.
\end{frame}
%%%%%%%%%%% FIM SLIDE 26 %%%%%%%%%%%%%%%%%%%%%

%%%%%%%%%%% SLIDE 27 %%%%%%%%%%%%%%%%%%%%%%%%%
\begin{frame}[fragile]
	\frametitle{Acentos e cedilha no texto}
	\begin{center}\let\tt\ttfamily
		\begin{tabular}{*7c}
			ò & ó & ô & ö & õ & ç & Ç \\
			\LCmdArg{`}{o} &\LCmdArg{'}{o} &\LCmdArg{\textasciicircum}{o} &\LCmdArg{\string"}{o} &						\LCmdArg{\textasciitilde}{o} &\LCmdArg{c}{c} &\LCmdArg{c}{C}
		\end{tabular}
	\end{center}
\end{frame}
%%%%%%%%%%% FIM SLIDE 27 %%%%%%%%%%%%%%%%%%%%%

%%%%%%%%%%% SLIDE 28 %%%%%%%%%%%%%%%%%%%%%%%%%
\begin{frame}{Conversão automática dos acentos}
	O pacote \Lsty{inputenc} faz internamente a conversão automática dos acentos e o usuário não tem de preocupar-se com os comandos de acentuação:
	\pause
	\begin{center}
		á $\longrightarrow$ \texttt{\LCmd{'}a}
	\end{center}
\pause
	No entanto, se não existirem recursos no teclado de sua máquina para acentuar, você ainda poderá acentuar seu texto usando os comandos.
\end{frame}
%%%%%%%%%%% FIM SLIDE 28 %%%%%%%%%%%%%%%%%%%%%

%%%%%%%%%%% SLIDE 29 %%%%%%%%%%%%%%%%%%%%%%%%%
\begin{frame}{Especificação das línguas usadas no documento}
	\begin{itemize}
		\item O pacote \Lsty{babel} especifica as línguas usadas no documento (\Lopt{brazilian}, \Lopt{english}, etc.), definindo, entre outras coisas, as regras de hifenação (separação silábica);
		\pause
		\item A última língua especificada entre as opções é a língua geral do documento. \pause Exemplo:
			\begin{Codigo}{Especificação das línguas do documento}
				\LOA usepackage[italian, english, brazilian]{babel}
			\end{Codigo}
		e a língua geral do documento é o português do Brasil.
	\end{itemize}
\end{frame}
%%%%%%%%%%% FIM SLIDE 29 %%%%%%%%%%%%%%%%%%%%%

%%%%%%%%%%% SLIDE 30 %%%%%%%%%%%%%%%%%%%%%%%%%
\begin{frame}{Seleção das línguas do documento}
	\begin{itemize}
		\item O documento pode ser composto somente nas línguas especificadas no pacote \Lsty{babel}; 
		\pause
		\item A distribuição \TeX\ Live possui suporte para quase $50$ línguas;
		\pause
		\item Isso implica que o \LaTeX\ muda as palavras como ``Capítulo'', por exemplo, em ``Chapter'', dependendo da língua escolhida.
		\pause
		\item Pode-se compor um trecho de texto em inglês, em um documento em português, com:
		\pause
			\begin{Codigo}{Seleção local da língua}
				\LCmdArg{begin}{otherlanguage}\Larg{english}\n
				English text\n
				\LCmdArg{end}{otherlanguage}\n
				ou\n
				texto em português \LCmdArg{foreignlanguage}{english}\Larg{English text}
				continuando em português \dots
			\end{Codigo}
	\end{itemize}
\end{frame}
%%%%%%%%%%% FIM SLIDE 30 %%%%%%%%%%%%%%%%%%%%%

%%%%%%%%%%% SLIDE 31 %%%%%%%%%%%%%%%%%%%%%%%%%
\begin{frame}{Hifenação (divisão silábica)}
	A hifenação é feita automaticamente no \LaTeX, desde que o pacote \Lsty{babel} tenha sido carregado. No caso de ocorrer uma hifenação incorreta, a correção é feita usando-se:
	\pause
	\begin{Codigo}{Hifenação irregular}
		\LCmdArg{hyphenation}{PYTHON com-pu-ta-dor} \% (usado na área\n \% de declarações/correção global)\nn
		com\LCmd{-}pu\LCmd{-}ta\LCmd{-}ção \% (usado no corpo do texto/local)
	\end{Codigo}
\end{frame}
%%%%%%%%%%% FIM SLIDE 31 %%%%%%%%%%%%%%%%%%%%%

%%%%%%%%%%% SLIDE 32 %%%%%%%%%%%%%%%%%%%%%%%%%
\begin{frame}{Produzindo texto}
	\begin{itemize}
		\item Aspas: Não use \texttt{\string"\dots\string"}; use \texttt{`{}`\dots'{}'} que produz ``\dots'';
		\pause
		\item Apóstrofes: \texttt{d'alembertiano} produz d'alembertiano;
		\pause
		\item Hífens:
			\begin{center}\let\tt\ttfamily
				\begin{tabular}{ll}
					\tt madeira-branca & madeira-branca \\
					\pause					
					\tt linhas 117-{}-138 & linhas 117--138 \\
					\pause
					\tt verdadeiro-{}-{}-ou falso? & verdadeiro---ou falso? \\
					\pause
					\tt \$-3.2\$ & $-3.2$
				\end{tabular}
			\end{center}
	\end{itemize}
\end{frame}
%%%%%%%%%%% FIM SLIDE 32 %%%%%%%%%%%%%%%%%%%%%

%%%%%%%%%%% SLIDE 33 %%%%%%%%%%%%%%%%%%%%%%%%%
\begin{frame}{Reticências}
	\begin{itemize}
		\item Para exprimir uma reticência no texto, usa-se \LCmd{dots};
		\pause
		\item Note a diferença entre \texttt{...} que produz ... e \texttt{\string\dots} que produz \dots;
		\pause
		\item Três pontinhos não são adequados pois são interpretados como três sentenças vazias;
		\pause
		\item Na matemática existem várias reticências; na linha da base, no meio da linha, e vertical e diagonal nas matrizes:
		\pause
			\begin{center}
				\begin{tabular}{ll}
					\ldots 		& \LCmd{ldots} \\
					\pause
					\vdots 		& \LCmd{vdots} \\
					\pause
					$\ddots$ 	& \texttt{\$\string\ddots\$}\\
					\pause
					$a,\dots,z$	& \texttt{\$a, \string\ldots, z\$} ou \texttt{\$a, \string\dots, z\$} \\
					\pause
					$a+\dots+ z$	& \texttt{\$a+ \string\cdots+ z\$} ou \texttt{\$a+ \string\dots+ z\$} \\
				\end{tabular}
			\end{center}
			\pause
		\item \LCmd{dots} sempre produz a reticência adequada pelo contexto.
	\end{itemize}
\end{frame}
%%%%%%%%%%% FIM SLIDE 33 %%%%%%%%%%%%%%%%%%%%%

%%%%%%%%%%% SLIDE 34 %%%%%%%%%%%%%%%%%%%%%%%%%
\begin{frame}{Ligaduras}
	\begin{itemize}
		\item As ligaduras mais frequentes são:
\pause
		\medskip

		 ff fi fl ffi \ldots ao invés de f{}f f{}i f{}l f{}f{}i;
		 \pause
		\item Para evitar use-se um grupo vazio: \texttt{f\{\}f} que produz f{}f.
	\end{itemize}

	\bigskip

	\pause
	\begin{Resultado}{Usando a lupa}
		{\Huge ff fi fl ffi} \ldots ao invés de {\Huge f{}f f{}i f\mbox{}l f{}f{}i}.
	\end{Resultado}
\end{frame}
%%%%%%%%%%% FIM SLIDE 34 %%%%%%%%%%%%%%%%%%%%%

%%%%%%%%%%% SLIDE 35 %%%%%%%%%%%%%%%%%%%%%%%%%
\begin{frame}{Mudando o estilo do texto}
	\let\tt\ttfamily
		\begin{tabular}{lll}
			& Comando		& Declaração \\
			\textbf{Bold} & \tt\string\textbf\{\dots\}	& \tt\{\string\bfseries \dots\} \\
			\pause
			\texttt{Máquina de escrever} & \tt\string\texttt\{\dots\} & \tt\{\string\ttfamily \dots \}\\
			\pause
			\textit{Itálico} & \tt\string\textit\{\dots\}	& \tt\{\string\itshape\dots\} \\
			\pause
			\textsf{Sans serif} & \tt\string\textsf\{\dots\}& \tt\{\string\sffamily\dots\} \\
			\pause
			\textsc{Small Caps}	& \tt\string\textsc\{\dots\}& \tt\{\string\scshape\dots\} \\
			\pause
			\emph{Ênfase}  & \tt\string\emph\{\dots\} & \tt\{\string\em \dots\}
		\end{tabular}

		\pause
		\begin{itemize}
			\item Deve-se observar que o ênfase não usa sublinhado\footnote{O sublinhado jamais é usado em tipografia.}, e é obtido com itálico se o texto é normal e normal se o texto é itálico;
			\pause
			\item Os comandos produzem seu efeito somente sobre seu argumento (escopo);
			\pause
			\item Comandos e/ou declarações podem ser acumulados: \\ \texttt{\string\textbf\{\string					\itshape\ Itálico negro\}} produz \textbf{\itshape Itálico negro}.
		\end{itemize}
\end{frame}
%%%%%%%%%%% FIM SLIDE 35 %%%%%%%%%%%%%%%%%%%%%

%%%%%%%%%%% SLIDE 36 %%%%%%%%%%%%%%%%%%%%%%%%%
\begin{frame}{Serifas}
	\begin{itemize}
		\pause
		\item As serifas são os pequenos traços ou hastes que ocorrem nos prolongamentos das letras;
		\pause
		\item Servem para guiar o olhar ao longo do texto;
		\pause
		\item As serifas na base das letras formam uma linha que serve como referência para o olho ``trafegar'' na linha de texto (como um trem no trilho);
		\pause
		\item Ela aumenta a legibilidade do corpo do texto\footnote{Jamais se usa fonte \emph{sans serif} no corpo do texto.}.

		\pause
		\begin{Resultado}{Comparação}
			\centering{\Large\string_\string_\textrm{Com serifa}\string_\string_ \hspace{2cm} \string_\string_\textsf{Sem serifa}\string_\string_}
		\end{Resultado}
	\end{itemize}
\end{frame}
%%%%%%%%%%% FIM SLIDE 36 %%%%%%%%%%%%%%%%%%%%%

%%%%%%%%%%% SLIDE 37 %%%%%%%%%%%%%%%%%%%%%%%%%
\begin{frame}{Mudando o tamanho dos fontes}
	\let\tt\ttfamily
	\begin{center}
		\begin{tabular}{ll}
			\pause
			\tiny tiny & \tt\{\string\tiny\ \dots\} \\
			\pause
			\scriptsize scriptsize & \tt\{\string\scriptsize\ \dots\} \\
			\pause
			\footnotesize footnotesize & \tt\{\string\footnotesize\ \dots\} \\
			\pause
			\small small & \tt\{\string\small\ \dots\} \\
			\pause
			\normalsize normalsize &\tt\{\string\normalsize\ \dots\} \\
			\pause
			\large\strut large & \tt\{\string\large\ \dots\} \\
			\pause
			\Large\strut Large & \tt\{\string\Large\ \dots\} \\
			\pause
			\LARGE\strut LARGE & \tt\{\string\LARGE\ \dots\} \\
			\pause
			\huge\strut huge & \tt\{\string\huge\ \dots\} \\
			\pause
			\Huge\strut Huge & \tt\{\string\Huge\ \dots\}
		\end{tabular}
	\end{center}

	\pause
	\begin{Observacao}{}
		Escopo da definição delimitado pelo grupo.
	\end{Observacao}
\end{frame}
%%%%%%%%%%% FIM SLIDE 37 %%%%%%%%%%%%%%%%%%%%%

%%%%%%%%%%% SLIDE 38 %%%%%%%%%%%%%%%%%%%%%%%%%
\begin{frame}{Alinhamento do texto}
	Ambientes \emph{center}, \emph{flushleft} e \emph{flushright}:

	\begin{center}
		\Huge Centrado
	\end{center}
	\pause

	\begin{flushleft}
		\Huge Esquerda
	\end{flushleft}
	\pause

	\begin{flushright}
		\Huge Direita
	\end{flushright}
\end{frame}
%%%%%%%%%%% FIM SLIDE 38 %%%%%%%%%%%%%%%%%%%%%

%%%%%%%%%%% SLIDE 39 %%%%%%%%%%%%%%%%%%%%%%%%%
\begin{frame}{Quebra de linha, parágrafo e página}
	\begin{itemize}
		\item Quebra de linha: \texttt{\string\\ } ou \texttt{\string\newline};
		\pause
		\item Quebra de página: \texttt{\string\newpage}.
	\end{itemize}
\end{frame}
%%%%%%%%%%% FIM SLIDE 39 %%%%%%%%%%%%%%%%%%%%%

%%%%%%%%%%% SLIDE 40 %%%%%%%%%%%%%%%%%%%%%%%%%
\begin{frame}{Notas de rodapé}
	As notas de rodapé podem ser obtidas colocando-se, no lugar do texto onde deve ser referenciada a nota, o comando \LCmdArg{footnote}{Texto da nota}, tendo como argumento o texto da nota. 
	
	\pause
	\begin{Codigo}{Exemplo}
		42\LCmdArg{footnote}{A resposta para a vida o universo e tudo mais}
	\end{Codigo}

	\pause
	Produz a saída:
	\begin{Resultado}{}
		42\footnote{A resposta para a vida o universo e tudo mais}
	\end{Resultado}
\end{frame}
%%%%%%%%%%% FIM SLIDE 40 %%%%%%%%%%%%%%%%%%%%%

%%%%%%%%%%% SLIDE 41 %%%%%%%%%%%%%%%%%%%%%%%%%
\begin{frame}{Produzindo títulos de trabalhos}
	\begin{Codigo}{Declarações}
		\LCmdArg{title}{Título}\n
		\LCmdArg{author}{Autor}\n
		\LCmdArg{date}{Data} ou \LCmdArg{date}{}
	\end{Codigo}

\pause
	Observações:
	\begin{itemize}
		\item \texttt{\string\date\{\}} omite a data do documento;
		\pause
		\item Omitindo-se o comando \texttt{\string\date}, é tomada a data corrente da máquina.
	\end{itemize}

\pause
	\begin{Codigo}{Produzindo}
		\LCmd{maketitle}
	\end{Codigo}
\end{frame}
%%%%%%%%%%% FIM SLIDE 41 %%%%%%%%%%%%%%%%%%%%%

%%%%%%%%%%% SLIDE 42 %%%%%%%%%%%%%%%%%%%%%%%%%
\begin{frame}{Exemplo de uso de título de trabalho}
	\begin{Codigo}{Estrutura no fonte}
		\LCmdArg{documentclass}{book}\n
		\LCmdArg{title}{O Guia do Mochileiro das Galáxias}\n
		\LCmdArg{author}{Douglas Adams}\n
		\LCmdArg{date}{}\n
		\LCmdArg{begin}{document}\n
		\LCmd{maketitle}\n

		O Universo é tão grande que comparado a ele mesmo ele é infinitamente menor...\n
		\LCmdArg{end}{document}
	\end{Codigo}
\end{frame}
%%%%%%%%%%% FIM SLIDE 42 %%%%%%%%%%%%%%%%%%%%%

%%%%%%%%%%% SLIDE 43 %%%%%%%%%%%%%%%%%%%%%%%%%
\begin{frame}{Resultado da composição do título}
	\begin{Resultado}{Estrutura produzida}
		{\centering
		{\Large O Guia do Mochileiro das Galáxias}\\[\baselineskip]
		{Douglas Adams}\par}

		\vspace{4\baselineskip}

		O Universo é tão grande que comparado a ele mesmo ele é infinitamente menor...
	\end{Resultado}
\end{frame}
%%%%%%%%%%% FIM SLIDE 43 %%%%%%%%%%%%%%%%%%%%%

%%%%%%%%%%%%%%%%%%%%%%%%%% FIM AULA 01 %%%%%%%%%%%%%%%%%%%%%%%%%%